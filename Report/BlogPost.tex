\documentclass[twoside]{article}
\usepackage{fullpage}
\usepackage[pdftex]{graphicx}
\usepackage{wrapfig}
\usepackage{amsmath}
\usepackage{hyperref}
\usepackage{sectsty}
%\sectionfont{\fontsize{13}{15}\selectfont}
\usepackage{fancyhdr}
\usepackage{listings}
\usepackage{graphicx}
\usepackage{lstmisc}
\usepackage{xcolor}
\pagestyle{fancy}
\fancyhead{}
\fancyfoot{}
\renewcommand{\headrulewidth}{0pt}
\fancyfoot[L]{\emph{Konicki - CSI 370 - Final Report}}
\fancyfoot[R] {\thepage}
\newenvironment{code}{\fontfamily{lmtt}\selectfont}{}
\date{}

\definecolor{commentgreen}{rgb}{0.0, 0.500, 0.015}

\lstset{frame=tb,
    language=C++,
    basicstyle=\ttfamily,
    keywordstyle=\color{blue}\ttfamily,
    stringstyle=\color{red}\ttfamily,
    commentstyle=\color{commentgreen}\ttfamily,
    morecomment=[l][\color{magenta}]{\#},
    breaklines=true,
}

\lstset{frame=tb,
    language=MASM,
    basicstyle=\ttfamily,
    keywordstyle=\color{blue}\ttfamily,
    stringstyle=\color{red}\ttfamily,
    commentstyle=\color{commentgreen}\ttfamily,
    morecomment=[l][\color{magenta}]{\#},
    breaklines=true,
}

% Document
\begin{document}

    \title{GPR 340 - Artificial Intelligence for Games \\ Genetic Algorithm }
    \author{Anne Konicki \\ Champlain College \\ anne.konicki@mymail.champlain.edu \\ December 2024 }
    \maketitle


    \section{What is the Genetic Algorithm???}\label{sec:what-is-the-genetic-algorithm}
    The genetic algorithm is a machine learning algorithm that trains a model to complete a specific task as efficiently as possible.
    The algorithm takes a scenario, as well as a large assortment of agents.
    The agents will begin acting completely at random.
    After a short trial, each agent will be scored based on a programmer-defined ``fitness'' function.
    Based on this function, the agents that performed the best will be cloned.

    \bigbreak
    \noindent
    In subsequent runs, each cloned agent will have ``genetic mutations,'' or small modifications to the instructions in hopes of improving.
    The best agent from the previous generation will also be directly inserted into the next generation to ensure that it is not possible for the entire generation to mutate negatively.
    The process of running a generation, running the fitness function, cloning the best agents in a generation, and then mutating them will then repeat endlessly.
    Eventually, one of the agents will complete the task, at which point the agents will begin to optimize the task.

    \bigbreak
    \noindent
    Once the goal has been completed, or the programmer ends the project, the ideal is that at least one of the agents was able to complete the task efficiently.

    \section{So why did I want to do this???}\label{sec:so-why-did-i-want-to-do-this???}
    Machine Learning in video games has been something I've been interested in for years now.


\end{document}